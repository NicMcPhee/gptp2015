%%%%%%%%%%%%%%%%%%%% author.tex %%%%%%%%%%%%%%%%%%%%%%%%%%%%%%%%%%%
%
%%%%%%%%%%%%%%%% Springer %%%%%%%%%%%%%%%%%%%%%%%%%%%%%%%%%%

\title*{GP As If You Meant It: Real and Imaginary User Experience}
\titlerunning{GP As If You Meant It}

% your contribution title if the original one is too long
\author{William A. Tozier}
\authorrunning{Tozier}

%\institute{William A. Tozier}

\maketitle

\abstract{In this contribution I describe a \emph{kata}, or exercise intended for advanced GP users, called ``GP as if you meant it''. In the exercise, the human participant(s) are charged with trying to ``rescue'' an ineffectual but unstoppable GP system which is set up initially to only use ``random guessing''. The exercise is in the form of a game of alternating turns, in which the human ``User'' player is provided complete information but only a very limited toolkit, and the (automated) GP System ``player'' can only be modified and ``adjusted'' by adding new search operators and objectives. The crux of the exercise is the human players' development of cogent \emph{warrants} for every design decision they make when building and using GP systems---decisions which otherwise can be habitual or arbitrary. By explicitly prohibiting the most common response of  ``stop it and restart with different parameters'', the participants are mindful of ways to \emph{accommodate} the noted ``pathologies'' and ``symptoms''. The computational participant tends to just get warmer.}

\begin{keywords}
keywords to your chapter, these words should also be indexed
\end{keywords}
\index{keywords to your chapter}
\index{these words should also be indexed}
\\

[version of \today\ at \currenttime]

\subimport{../../../markdown/}{manuscript}




\bibliographystyle{spbasic}
\bibliography{gp-bibliography,tozier}
